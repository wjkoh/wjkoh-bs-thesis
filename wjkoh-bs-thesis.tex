%\documentclass[12pt,a4paper,twoside,draft,openright]{report}
\documentclass[12pt,a4paper,oneside,draft]{report}

% Palatino
\usepackage[T1]{fontenc}
\usepackage[sc]{mathpazo}

% 한글 설정 시작
\usepackage[hangul,nonfrench,finemath]{kotex}
\usepackage[default]{dhucs-interword}
\usehangulfontspec{ut}
\usepackage[hangul]{dhucs-setspace}
\usepackage{dhucs-gremph}

\usepackage{ifpdf}
\ifpdf
  \usepackage[unicode,pdftex,colorlinks]{hyperref}
  \input glyphtounicode\pdfgentounicode=1
\else
  \usepackage[unicode,dvipdfm,colorlinks]{hyperref}
\fi
% 한글 설정 끝

% Line spacing
\usepackage{setspace}
%\doublespacing
\onehalfspacing

% Header
\pagestyle{headings}

% Bibliography
\usepackage[autostyle=true]{csquotes}
\usepackage[backend=biber,style=ieee,natbib=true,hyperref=true]{biblatex}
\addbibresource{wjkoh-bs-thesis.bib}
\newcommand{\Kim}{\cite{Kim:2009:SMM:1531326.1531385}} 
\newcommand{\KimKwon}{\cite{Kim:2009:SMM:1531326.1531385,Kwon:2008:GME:1360612.1360679}}
\newcommand{\Igarashi}{\cite{Igarashi:2005:ASM:1073204.1073323}} 
\newcommand{\Floater}{\cite{Floater200319}} 
\newcommand{\Hormann}{\cite{Hormann:2006:MVC:1183287.1183295}}
\newcommand{\Lipman}{\cite{Lipman:2008:GC:1360612.1360677}}

% Math
\usepackage{amsmath}
\usepackage{amssymb}
%\usepackage{array}
\providecommand{\abs}[1]{\lvert#1\rvert}
\providecommand{\norm}[1]{\lVert#1\rVert}

\newcommand{\thesistitle}{케이지 기반의 대규모 운동 경로 편집법\\ (Cage-based Large-scale Motion Path Editing)}

\title{\thesistitle{}}
\author{고우종\\
서울대학교 컴퓨터공학부\\
\url{wjngkoh@gmail.com}
}
\date{\today}
%\keywords{Computer Graphics, Computer Animation,  Data-Driven Animation, Human Motion}

\begin{document}
% First cover
\begin{titlepage}
\begin{center}
% Upper part of the page
{\LARGE \thesistitle{}}\\[4.0cm]

% Supervisor
{\Large 지도교수 : 이제희}\\[1.5cm]

{\Large 이 논문을 공학학사 학위 논문으로 제출함.}\\[2.5cm]

% Bottom of the page
{\Large \today}\\[1.5cm]
\vfill

% Author
\Large
서울대학교 공과대학\\
컴 퓨 터 공 학 부\\
고 우 종\\[1.0cm]
{\Large 2012년 2월}
\end{center}
\end{titlepage}

% Another cover
\maketitle

\begin{abstract}
Your abstract goes here...
\end{abstract}

\tableofcontents

\chapter{서론}
최근 들어 매우 큰 규모의 군중이 등장하는 장면이 영화나 애니메이션, 비디오 게임
등에 자주 등장하고 있다. 이러한 군중 장면들은 보통 등장 인물들끼리 협동 혹은
적대하는 상호작용들 여러 개로 구성되어 있다. 예를 들면, 여러 명의 짐꾼이 같은
물건을 동시에 들어서 옮기는 경우나 아니면 여러 명의 병사들이 서로 전투를 하고
있는 장면을 들 수 있다. 이러한 상호 작용은 일종의 제약조건 (constraint)으로
작용하는데, 왜냐하면 상호작용에 참가하는 각 캐릭터들은 해당 상호작용에 참여하는
다른 캐릭터들과 상대적인 위치, 방향, 타이밍 면에서 잘 조정되어야 하기 때문이다.


캐릭터들의 운동 경로 (motion path)를 매우 복잡하게 상호적으로 종속하게 만든다.
왜냐하면, 상호작용이 어긋나지 않도록 유지하기 위해서는 각 캐릭터들의 매
프레임이 위치나 방향, 타이밍 면에서 다른 프레임과 맞도록 매우 미세하게
조정되어야 하기 때문이다.  더군다나, 이러한 거대한 군중 장면의 사용은 앞으로
자동화된 생성 방식들이 새로 개발됨에 따라 더욱 널리 사용되어질 것으로 예상된다.
따라서 이처럼 거대한 군중 장면을 상호작용들은 최대한 그대로 유지하면서
실시간으로 편집하는 새로운 방법이 애니메이터들에게 필요하게 되었다.


이 문제를 해결하기 위해, 본 논문에서는 문제를 처음부터 다시 재정의하고 새로운
접근 방식을 도입했다. 기존의 방식 \cite{Kim:2009:SMM:1531326.1531385}을
개선시키는 것은 기존 방식의 구조상 내재된 한계에 부딪혔다.  설명하자면, 이
문제의 가장 근본적인 어려움은 바로 두 개의 요구사항, 즉 속도와 정확도가 서로
상충하는, 역의 관계에 있다는 것이다. 우선, 앞서 언급한 이 제약조건들을 모두
동시에 만족시키기 위해서는 매우 정확하고 총체적인 계산이 필요하다.  기존의
방식들은 이 문제를 운동 경로의 모든 프레임들을 하나의 선형 시스템에 포함시킨 후
이 시스템을 풂으로써 해결하였다. 하지만, 질적인 품질을 높이는 대신에 이 문제의
계산 복잡도를 프레임의 총 개수에 의존하게 만듦으로써 확장 가능 (scalable)하지
않도록 만들었다. 따라서 프레임의 총 개수가 증가함에 따라 기존의 방식은
인터랙티브한 편집 속도를 달성하지 못하게 된다.

이 한계를 극복하기 위해서, 본 저자는 계산 복잡도가 프레임의 총 개수가 아닌,
고정된 개수의 제어점에만 의존하는 새로운 방법을 고안해냈다. 또한, 더 이상
애니메이터가 대규모의 군중 장면을 운동 경로 하나하나씩 편집할 수 없으므로,
조작용 핸들을 새로 정의하였다.


결과적으로, 본 논문에서는 새로운 대규모의 군중 장면을 편집하는 기술을 제시한다.
이 기술은 애니메이터들에게 원래의 상호작용은 최대한 보존하면서 운동 경로를
자유자재로 편집할 수 있도록 한다.


\chapter{관련 연구} \section{Mean-Value Coordinates} 본 논문에서는 무게중심좌표
(barycentric coordinates)의 일반화 중 하나인 중간값 좌표 (Mean-Value
Coordinates)를 사용하였다. 이 좌표는 조화함수를 위한 중간값 정리에서 착안하여
만들어진 좌표로서, 다각형의 꼭지점들에 매겨진 값들로부터 중간값들을 보간하는데
유용하다 \Floater.

무게중심좌표는 볼록다각형에는 적용할 수 있으나, 볼록하지 않은 다각형에 대해서는
일반적으로 적용할 수 없다.  반면에, 중간값 좌표는 모서리끼리 스스로 교차하지
않는, 임의의 평면다각형에 대해서도 잘 정의된다 \Hormann. 운동 경로들을 정교하게
덮기 위해서는 오목 껍데기 (concave hull)를 사용해야 하므로 중간값 좌표는 본
논문에서 다루는 문제에 사용하기에 적합하다.

%One generalization of barycentric coordinates we use is the Mean-Value
%Coordinates (MVC). The coordinates are initially motivated by the Mean Value
%Theorem for harmonic functions and useful for interpolating data that is given
%at the vertices of the polygons \Floater.
%
%Although barycentric coordinates also can be extended to convex polygons, they
%are usually unable to cover non-convex polygons. On the other hand, the
%Mean-Value Coordinates are well-defined for arbitrary planar polygons without
%self-intersections \Hormann. Thus, the coordinates are suitable to our setting
%because we need to use a concave hull to overlap motion paths precisely.

We explain the definition of the Mean-Value Coordinates briefly. Consider a
point $x \in \mathbb{R}^2$ and the simple polygon $[v_0, v_1, ..., v_k, v_{k+1}
= v_0]$ defined cyclically. Then, let $\alpha_i$ be the angle $\angle
v_{i+1},x,v_{i}$ in the polygon.

\begin{align}
\lambda_i = \frac{w_i}{\sum_{j=1}^{k}w_j}, \quad w_i =
2\frac{\tan(\alpha_{i-1}/2) + \tan(\alpha_{i}/2)}{\norm{x - v_i}}
\end{align}

The weights $(\lambda_0, ..., \lambda_k)$ are the Mean-Vale Coordinates for $x$
with respect to $v_0, ...  , v_k$.

\section{Green Coordinates}

\chapter{개관}

\chapter{결과}

\chapter{결론}

\printbibliography
\end{document}
